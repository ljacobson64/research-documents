%%%%%%%%%%%%%%%%%%%%%%%%%%%%%%%%%%%%%%%%%%%%%%%%%%%%%%%%%%%%%%%%%%%%%%%%%%%%%%%%
\section{Radiation Transport}
\label{sec:bg:rt}
%%%%%%%%%%%%%%%%%%%%%%%%%%%%%%%%%%%%%%%%%%%%%%%%%%%%%%%%%%%%%%%%%%%%%%%%%%%%%%%%

Much of the information in this section is paraphrased from Lewis \& Miller \cite{lewis_miller}.

%===============================================================================
\subsection{Transport Equation}
\label{sec:bg:rt:te}
%===============================================================================

%-------------------------------------------------------------------------------
\subsubsection{Particle Distributions}
\label{sec:bg:rt:te:pd}
%-------------------------------------------------------------------------------

In general, there are seven independent variables that are required to fully describe the distribution of neutrons: three coordinates specifying position ($\vec{r}$), two angles specifying the particle's direction of travel ($\hat{\Omega}$), the particle's energy ($E$), and time ($t$).

The most fundamental dependent variable in transport theory is the \textit{particle density distribution}.
The particle density distribution $N\left(\vec{r},\hat{\Omega},E,t\right)dEd\hat{\Omega}dV$ is defined as the number of particles in a volume element $dV$ about $\vec{r}$ traveling in the cone of directions $d\hat{\Omega}$ about $\hat{\Omega}$ with energies between $E$ and $E + dE$ at time $t$.

The particle density distribution is rarely directly considered in transport calculations, however.
The most common quantity considered is the \textit{angular flux}:
\begin{equation}\label{eq:bg:rt:angular-flux}
  \psi\left(\vec{r},\hat{\Omega},E,t\right) \equiv vN\left(\vec{r},\hat{\Omega},E,t\right)
\end{equation}
where $v$ is the particle speed.

It is sometimes useful to consider the flux independent of angle.
In these cases, the \textit{scalar flux} may be considered:
\begin{equation}\label{eq:bg:rt:scalar-flux}
  \phi\left(\vec{r},E,t\right) \equiv \int_{4\pi}\psi\left(\vec{r},\hat{\Omega},E,t\right)d\hat{\Omega}.
\end{equation}

The \textit{current vector}, which is a measure of the directionality of particles, is defined as
\begin{equation}\label{eq:bg:rt:current-vector}
  \vec{J}\left(\vec{r},E,t\right) \equiv \int_{4\pi}\psi\left(\vec{r},\hat{\Omega},E,t\right)\hat{\Omega} d\hat{\Omega}.
\end{equation}

The \textit{reaction rate} for reactions of type $x$ is
\begin{equation}\label{eq:bg:rt:rxn-rate-angular}
  R = \int_V\int_{4\pi}\int_0^\infty\sigma_x\left(\vec{r},\hat{\Omega},E\right)\psi\left(\vec{r},\hat{\Omega},E,t\right)dEd\hat{\Omega}dV
\end{equation}
where $\sigma_x$ is the cross section for reaction $x$.
Cross sections are almost always independent of the particle direction, so the reaction rate equation can be simplified as
\begin{equation}\label{eq:bg:rt:rxn-rate-scalar}
  R = \int_V\int_0^\infty\sigma_x\left(\vec{r},E\right)\phi\left(\vec{r},E,t\right)dEdV.
\end{equation}

If $\sigma_x$ refers to the total cross section $\sigma_t$, then the reaction rate is referred to as the \textit{collision density}.

%-------------------------------------------------------------------------------
\subsubsection{Transport Equation with General Source}
\label{sec:bg:rt:te:te}
%-------------------------------------------------------------------------------

The transport equation is a balance between all possible mechanisms for the number of particles in $dEd\hat{\Omega}dV$ to change.
There are three such mechanisms:
\begin{enumerate}
  \item net streaming out of $dV$, given by the streaming term $\hat{\Omega}\cdot\vec{\nabla}\psi\left(\vec{r},\hat{\Omega},E,t\right)$,
  \item collisions in $dV$ that result in particle absorption or scattering out of $dEdV$, given by the collision term $\sigma_t\left(\vec{r},E\right)\psi\left(\vec{r},\hat{\Omega},E,t\right)$, and
  \item emission of particles in $dV$ (includes external sources, scattering, and fission), given by a general source term $q\left(\vec{r},\hat{\Omega},E,t\right)$.
\end{enumerate}

The time-dependent transport equation is a balance between these three processes, and it is given by
\begin{equation}\label{eq:bg:rt:transport-totsrc-timedep}
  \frac{1}{v}\frac{\partial}{\partial t}\psi\left(\vec{r},\hat{\Omega},E,t\right) +
  \hat{\Omega}\cdot\vec{\nabla}\psi\left(\vec{r},\hat{\Omega},E,t\right) +
  \sigma_t\left(\vec{r},E\right)\psi\left(\vec{r},\hat{\Omega},E,t\right) =
  q\left(\vec{r},\hat{\Omega},E,t\right).
\end{equation}
If the dependence on time is neglected, the time-dependent term disappears. The time-independent transport equation is given by
\begin{equation}\label{eq:bg:rt:transport-totsrc}
  \hat{\Omega}\cdot\vec{\nabla}\psi\left(\vec{r},\hat{\Omega},E\right) +
  \sigma_t\left(\vec{r},E\right)\psi\left(\vec{r},\hat{\Omega},E\right) =
  q\left(\vec{r},\hat{\Omega},E\right).
\end{equation}

%-------------------------------------------------------------------------------
\subsubsection{Source Terms}
\label{sec:bg:rt:te:st}
%-------------------------------------------------------------------------------

The source term $q\left(\vec{r},\hat{\Omega},E,t\right)$ includes contributions from three sources: external sources, scattering, and fission.
It can be represented as a sum of these three contributions:
\begin{equation}\label{eq:bg:rt:source}
  q = q_{ex} + q_s + q_f.
\end{equation}

External sources are known distributions of source particles that are entirely independent of the flux.
External sources are represented as a known distribution $q_{ex}\left(\vec{r},\hat{\Omega},E,t\right)$.

Scattering sources are the result of particles scattering from one angle and energy to another.
Particles at the original angle and energy are removed from the system while particles at the new angle and energy are born.
The source due to scattering is given by the scattering term:
\begin{equation}\label{eq:bg:rt:scattering-term}
  q_s\left(\vec{r},\hat{\Omega},E,t\right) \equiv
  \int_{4\pi}\int_0^\infty\sigma_s\left(\vec{r},\hat{\Omega}^\prime\rightarrow\hat{\Omega},E^\prime\rightarrow E\right)\psi\left(\vec{r},\hat{\Omega}^\prime,E^\prime,t\right)dE^\prime d\hat{\Omega}^\prime
\end{equation}
where the scattering cross section $\sigma_s\left(\vec{r},\hat{\Omega}^\prime\rightarrow\hat{\Omega},E^\prime\rightarrow E\right)$ is the cross section for particles to scatter from $\hat{\Omega}^\prime$ into $\hat{\Omega}$ and $E^\prime$ into $E$.

Fission sources only occur if a multiplying medium exists in the system.
Fission sources will not be considered in this work.

%-------------------------------------------------------------------------------
\subsubsection{Transport Equation with Explicit Scattering Source}
\label{sec:bg:rt:te:te2}
%-------------------------------------------------------------------------------

Separating the total source term in the time-independent transport equation into its external and scattering components yields
\begin{multline}\label{eq:bg:rt:transport-timedep}
  \frac{1}{v}\frac{\partial}{\partial t}\psi\left(\vec{r},\hat{\Omega},E,t\right) +
  \hat{\Omega}\cdot\vec{\nabla}\psi\left(\vec{r},\hat{\Omega},E,t\right) +
  \sigma_t\left(\vec{r},E\right)\psi\left(\vec{r},\hat{\Omega},E,t\right) = \\
  \int_{4\pi}\int_0^\infty\sigma_s\left(\vec{r},\hat{\Omega}^\prime\rightarrow\hat{\Omega},E^\prime\rightarrow E\right)\psi\left(\vec{r},\hat{\Omega}^\prime,E^\prime,t\right)dE^\prime d\hat{\Omega}^\prime +
  q\left(\vec{r},\hat{\Omega},E,t\right).
\end{multline}
The time-independent version is given by 
\begin{multline}\label{eq:bg:rt:transport}
  \hat{\Omega}\cdot\vec{\nabla}\psi\left(\vec{r},\hat{\Omega},E\right) +
  \sigma_t\left(\vec{r},E\right)\psi\left(\vec{r},\hat{\Omega},E\right) = \\
  \int_{4\pi}\int_0^\infty\sigma_s\left(\vec{r},\hat{\Omega}^\prime\rightarrow\hat{\Omega},E^\prime\rightarrow E\right)\psi\left(\vec{r},\hat{\Omega}^\prime,E^\prime\right)dE^\prime d\hat{\Omega}^\prime +
  q\left(\vec{r},\hat{\Omega},E\right).
\end{multline}

%-------------------------------------------------------------------------------
\subsubsection{Operator Notation}
\label{sec:bg:rt:te:on}
%-------------------------------------------------------------------------------

It is often useful to write the transport equation in operator notation.
In this case, the time-independent \textit{transport operator} $H$ operates on the angular flux $\psi$ and is given by
\begin{multline}\label{eq:bg:rt:transport-operator}
  H\psi \equiv
  \hat{\Omega}\cdot\vec{\nabla}\psi\left(\vec{r},\hat{\Omega},E\right) +
  \sigma_t\left(\vec{r},E\right)\psi\left(\vec{r},\hat{\Omega},E\right) - \\
  \int_{4\pi}\int_0^\infty\sigma_s\left(\vec{r},\hat{\Omega}^\prime\rightarrow\hat{\Omega},E^\prime\rightarrow E\right)\psi\left(\vec{r},\hat{\Omega}^\prime,E^\prime\right)dE^\prime d\hat{\Omega}^\prime.
\end{multline}
The transport equation in operator notation is then simply
\begin{equation}\label{eq:bg:rt:transport-opnot}
  H\psi = q.
\end{equation}

%===============================================================================
\subsection{Adjoint Transport Equation}
\label{sec:bg:rt:ate}
%===============================================================================
