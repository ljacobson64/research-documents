%%%%%%%%%%%%%%%%%%%%%%%%%%%%%%%%%%%%%%%%%%%%%%%%%%%%%%%%%%%%%%%%%%%%%%%%%%%%%%%%
\section{Radiation Transport}
\label{sec:bg:rt}
%%%%%%%%%%%%%%%%%%%%%%%%%%%%%%%%%%%%%%%%%%%%%%%%%%%%%%%%%%%%%%%%%%%%%%%%%%%%%%%%

Placeholder

%===============================================================================
\subsection{Transport Equation}
\label{sec:bg:rt:te}
%===============================================================================

In transport calculations, neutrons have four distinguishing characteristics: position ($\vec{r}$), direction ($\hat{\Omega}$), energy ($E$), and time ($t$).
There are three components of position, two components of direction, and one component of energy and time, for a total of seven components.
Any neutrons that have the same values of these seven components are indistinguishable.

These seven components form a seven-dimensional phase space.
The quantity that describes how the neutrons are distributed within this phase space is referred to as the angular flux $\psi\left(\vec{r},\hat{\Omega},E,t\right)$.
The angular flux defines the number of neutrons that exist at point $\vec{r}$, going in direction ${\hat{\Omega}}$, with energy $E$, at time $t$.

The angular flux can be calculated by solving the transport equation, which is a balance equation that conserves neutrons.
Each term in the equation represents a net gain or loss of neutrons in all regions of phase space.
The transport equation is given by

\begin{multline}
  \frac{1}{v}\frac{\partial}{\partial t}\psi\left(\vec{r},\hat{\Omega},E,t\right) +
  \hat{\Omega}\cdot\vec{\nabla}\psi\left(\vec{r},\hat{\Omega},E,t\right) +
  \sigma_t\left(\vec{r},E\right)\psi\left(\vec{r},\hat{\Omega},E,t\right) - \\
  \iint\sigma_s\left(\vec{r},\hat{\Omega}^\prime\rightarrow\hat{\Omega},E^\prime\rightarrow E\right)\psi\left(\vec{r},\hat{\Omega}^\prime,E^\prime,t\right)d\hat{\Omega}^\prime dE^\prime =
  q\left(\vec{r},\hat{\Omega},E,t\right).
\end{multline}

The streaming term $\hat{\Omega}\cdot\vec{\nabla}\psi\left(\vec{r},\hat{\Omega},E,t\right)$
represents the net removal of neutrons in a region of phase space due to streaming in the $\hat{\Omega}$ direction.

The collision term $\sigma_t\left(\vec{r},E\right)\psi\left(\vec{r},\hat{\Omega},E,t\right)$
represents the net removal of neutrons in a region of phase space due to them undergoing collisions.
The quantity $\sigma_t\left(\vec{r},E\right)$ is the total cross section, which is a measure of the probability that a neutron will undergo any possible interaction.

The scattering term $\iint\sigma_s\left(\vec{r},\hat{\Omega}^\prime\rightarrow\hat{\Omega},E^\prime\rightarrow E\right)\psi\left(\vec{r},\hat{\Omega}^\prime,E^\prime,t\right)d\hat{\Omega}^\prime dE^\prime$
represents the net gain of neutrons in a region of phase space due to scattering.
The quantity $\sigma_s\left(\vec{r},\hat{\Omega}^\prime\rightarrow\hat{\Omega},E^\prime\rightarrow E\right)$ is the scattering cross section,
which is a measure of the probability of a neutron scattering from angle $\hat{\Omega}^\prime$ into angle ${\hat{\Omega}}$ and energy $E^\prime$ into energy $E$.
The term is integrated over $\hat{\Omega}^\prime$ and $E^\prime$ to result in a quantity that includes contributions from all possible starting angles and energies.

The source term $q\left(\vec{r},\hat{\Omega},E,t\right)$
represents the gain of neutrons in a region of phase space from external sources.
It is usually a prescribed quantity.

In problems with no time dependence, the time-dependent term can be removed from the transport equation,
and the angular flux is represented in six-dimensional phase space as $\psi\left(\vec{r},\hat{\Omega},E\right)$.
The time-independent transport equation is given by

\begin{multline}\label{eq:bg:transport}
  \hat{\Omega}\cdot\vec{\nabla}\psi\left(\vec{r},\hat{\Omega},E\right) +
  \sigma_t\left(\vec{r},E\right)\psi\left(\vec{r},\hat{\Omega},E\right) - \\
  \iint\sigma_s\left(\vec{r},\hat{\Omega}^\prime\rightarrow\hat{\Omega},E^\prime\rightarrow E\right)\psi\left(\vec{r},\hat{\Omega}^\prime,E^\prime\right)d\hat{\Omega}^\prime dE^\prime =
  q\left(\vec{r},\hat{\Omega},E\right).
\end{multline}

% TO REFERENCE: \ref{eq:bg:transport}

The transport equation is sometimes written in operator notation, where the streaming, collision, and scattering terms are all combined into a single term $H$.
The transport equation is written in operator notation simply as

\begin{equation}
  H\psi = q.
\end{equation}

There are two main classes of methods for solving the transport equation: deterministic methods and Monte Carlo methods.
These methods will be explored in Sections \ref{sec:bg:rt:determ} and \ref{sec:bg:rt:determ}.

%===============================================================================
\subsection{Adjoint Transport Equation}
\label{sec:bg:rt:determ}
%===============================================================================

Placeholder

%===============================================================================
\subsection{Deterministic Radiation Transport}
\label{sec:bg:rt:determ}
%===============================================================================

Placeholder

%===============================================================================
\subsection{Monte Carlo Radiation Transport}
\label{sec:bg:rt:mc}
%===============================================================================

Placeholder

Reference to \ac{mcnp} \cite{mcnp5-theory}.

%===============================================================================
\subsection{Variance Reduction and Hybrid Methods}
\label{sec:bg:rt:vr}
%===============================================================================

Placeholder
