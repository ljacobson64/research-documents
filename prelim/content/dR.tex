\chapter{Perturbation in the Detector Response}
\label{chap:dr}

Section \ref{sec:bg:rt} introduced the transport equation and the concept of adjoint transport.
Section \ref{sec:bg:pert} showed Greenspan's derivation of an equation for the perturbation in the detector response in operator notation.
In this section, an equation for the perturbation in the detector response in a form suitable for use by deterministic transport codes will be derived.

%%%%%%%%%%%%%%%%%%%%%%%%%%%%%%%%%%%%%%%%%%%%%%%%%%%%%%%%%%%%%%%%%%%%%%%%%%%%%%%%
\section{Perturbed Transport Equation}
\label{sec:dr:pte}
%%%%%%%%%%%%%%%%%%%%%%%%%%%%%%%%%%%%%%%%%%%%%%%%%%%%%%%%%%%%%%%%%%%%%%%%%%%%%%%%

A perturbed version of the transport equation is derived in this section.

The perturbed total cross section $\bar{\sigma}_t$, the perturbed scattering cross section $\bar{\sigma}_s$, and the angular flux in the perturbed system $\bar{\psi}$ are defined as follows:
\begin{equation}\label{eq:dr:sigma_t_bar}
  \bar{\sigma}_t\left(\vec{r},E\right) \equiv
  \sigma_t      \left(\vec{r},E\right) +
  \delta\sigma_t\left(\vec{r},E\right)
\end{equation}
\begin{multline}\label{eq:dr:sigma_s_bar}
  \bar{\sigma}_s\left(\vec{r},\hat{\Omega}^\prime\rightarrow\hat{\Omega},E^\prime\rightarrow E\right) \equiv \\
  \sigma_s      \left(\vec{r},\hat{\Omega}^\prime\rightarrow\hat{\Omega},E^\prime\rightarrow E\right) +
  \delta\sigma_s\left(\vec{r},\hat{\Omega}^\prime\rightarrow\hat{\Omega},E^\prime\rightarrow E\right)
\end{multline}
\begin{equation}\label{eq:dr:psi_bar}
  \bar{\psi}\left(\vec{r},\hat{\Omega},E\right) \equiv
  \psi      \left(\vec{r},\hat{\Omega},E\right) +
  \delta\psi\left(\vec{r},\hat{\Omega},E\right).
\end{equation}

The transport equation is repeated here:
\begin{multline}\label{eq:dr:te}
  \hat{\Omega}\cdot\vec{\nabla}\psi\left(\vec{r},\hat{\Omega},E\right) +
  \sigma_t\left(\vec{r},E\right)\psi\left(\vec{r},\hat{\Omega},E\right) = \\
  \int_{4\pi}\int_0^\infty\sigma_s\left(\vec{r},\hat{\Omega}^\prime\rightarrow\hat{\Omega},E^\prime\rightarrow E\right)\psi\left(\vec{r},\hat{\Omega}^\prime,E^\prime\right)dE^\prime d\hat{\Omega}^\prime +
  q\left(\vec{r},\hat{\Omega},E\right)
\end{multline}
or, in simpler terms,
\begin{equation}\label{eq:dr:te_simple}
  \hat{\Omega}\cdot\vec{\nabla}\psi +
  \sigma_t\psi -
  \iint\sigma_s\psi d\hat{\Omega}^\prime dE^\prime =
  q.
\end{equation}
The transport equation in perturbed system is given by
\begin{equation}\label{eq:dr:te_pert}
  \hat{\Omega}\cdot\vec{\nabla}\bar{\psi} +
  \bar{\sigma}_t\bar{\psi} -
  \iint\bar{\sigma}_s\bar{\psi} d\hat{\Omega}^\prime dE^\prime =
  q.
\end{equation}
Expanding the perturbed terms yields
\begin{equation}\label{eq:dr:te_pert_2}
  \hat{\Omega}\cdot\vec{\nabla}\left(\psi + \delta\psi\right) +
  \left(\sigma_t + \delta\sigma_t\right)\left(\psi + \delta\psi\right) -
  \iint\left(\sigma_s + \delta\sigma_s\right)\left(\psi + \delta\psi\right)d\hat{\Omega}^\prime dE^\prime =
  q.
\end{equation}
Subtracting the original transport equation yields
\begin{equation}\label{eq:dr:te_pert_3}
  \hat{\Omega}\cdot\vec{\nabla}\delta\psi +
  \left(\sigma_t\delta\psi + \delta\sigma_t\psi + \delta\sigma_t\delta\psi\right) - \\
  \iint\left(\sigma_s\delta\psi + \delta\sigma_s\psi + \delta\sigma_s\delta\psi\right)d\hat{\Omega}^\prime dE^\prime =
  0.
\end{equation}
In perturbation theory, it is common to discard the second-order terms.
The result of doing so is given by
\begin{equation}\label{eq:dr:te_pert_4}
  \hat{\Omega}\cdot\vec{\nabla}\delta\psi +
  \left(\sigma_t\delta\psi + \delta\sigma_t\psi\right) -
  \iint\left(\sigma_s\delta\psi + \delta\sigma_s\psi\right)d\hat{\Omega}^\prime dE^\prime =
  0.
\end{equation}
Finally, after rearranging terms, the equation becomes
\begin{equation}\label{eq:dr:pte}
  \hat{\Omega}\cdot\vec{\nabla}\delta\psi +
  \sigma_t\delta\psi -
  \iint\sigma_s\delta\psi d\hat{\Omega}^\prime dE^\prime =
  \iint\delta\sigma_s\psi d\hat{\Omega}^\prime dE^\prime - \delta\sigma_t\psi.
\end{equation}
Equation \ref{eq:dr:pte} is written in operator notation as
\begin{equation}
  H\delta\psi = -\delta H\psi.
\end{equation}

Equation \ref{eq:dr:pte} is notable because of its similarity to the original transport equation.
The terms on the left represent the transport operator $H$ operating on the perturbation in the flux $\delta\psi$.
The terms on the right represent the perturbation in the transport operator $\delta H$ operating on the unperturbed flux $\psi$.
The terms on the right can then be thought of as a ``perturbed source term.''

Thus, the perturbation in the angular flux can be calculated by solving the transport equation with unperturbed cross sections and with the external source equal to $-\delta H\psi$.

%%%%%%%%%%%%%%%%%%%%%%%%%%%%%%%%%%%%%%%%%%%%%%%%%%%%%%%%%%%%%%%%%%%%%%%%%%%%%%%%
\section{Calculation of $\delta R$}
\label{sec:dr:dr}
%%%%%%%%%%%%%%%%%%%%%%%%%%%%%%%%%%%%%%%%%%%%%%%%%%%%%%%%%%%%%%%%%%%%%%%%%%%%%%%%
