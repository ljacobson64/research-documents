%%%%%%%%%%%%%%%%%%%%%%%%%%%%%%%%%%%%%%%%%%%%%%%%%%%%%%%%%%%%%%%%%%%%%%%%%%%%%%%%
\section{Perturbation Theory}
\label{sec:bg:pert}
%%%%%%%%%%%%%%%%%%%%%%%%%%%%%%%%%%%%%%%%%%%%%%%%%%%%%%%%%%%%%%%%%%%%%%%%%%%%%%%%

Perturbation theory is a mathematical technique that aims to find an approximate solution to a problem by first finding an exact solution to a related but simpler problem.
It is often used to solve problems that do not have an exact solution but are similar to a problems that do have an exact solution.
The technique involves breaking a problem into solvable and perturbed parts.

In radiation transport problems, perturbation theory has traditionally been used to estimate the effect of a small alteration in a fission reactor on its reactivity.
The idea of using perturbation theory to calculate parameters other than the reactivity, for nuclear systems that are not reactors, is a relatively more recent development.

Greenspan showed in 1974 that perturbation techniques can be used to calculate the effect of a perturbation on the flux distribution in inhomogeneous, or fixed-source, problems \cite{greenspan}. The following section describes his work.

%===============================================================================
\subsection{Detector Response in Inhomogeneous Systems}
\label{sec:bg:pert:greenspan}
%===============================================================================

The transport equation in operator notation is given by
\begin{equation}\label{eq:bg:pt:green29}
  H\psi = q
\end{equation}
where $H$ is the transport operator, $\psi$ is the angular flux, and $q$ is the fixed source.
Similarly, the adjoint transport equation in operator notation is given by
\begin{equation}\label{eq:bg:pt:green163}
  H^+\psi^+ = q^+
\end{equation}
where $H^+$ is the adjoint transport operator, $\psi^+$ is the adjoint angular flux, and $q^+$ is the adjoint source.

In the perturbed system, the forward and adjoint transport equations are given by
\begin{equation}\label{eq:bg:pt:green29_perturbed}
  \bar{H}\bar{\psi} = q
\end{equation}
and
\begin{equation}\label{eq:bg:pt:green163_perturbed}
  \bar{H}^+\bar{\psi}^+ = q^+,
\end{equation}
where $\bar{H}$ is the perturbed transport operator, $\bar{\psi}$ is the perturbed flux, $\bar{H}^+$ is the perturbed adjoint transport operator, and $\bar{\psi}^+$ is the perturbed adjoint flux.

Combining Equations \ref{eq:bg:pt:green29} and \ref{eq:bg:pt:green29_perturbed}, rearranging terms, and discarding the second-order term $\delta H\delta\psi$ yields the following identity:
\begin{equation}\label{eq:bg:pt:green44a}\begin{split}
  H\psi       & = \bar{H}\bar{\psi} \\
  H\psi       & = \left(H+\delta H\right)\left(\psi+\delta\psi\right) \\
  0           & = H\delta\psi + \delta H\psi + \cancel{\delta H\delta\psi} \\
  0           & = H\delta\psi + \delta H\psi \\
  H\delta\psi & = -\delta H\psi.
\end{split}\end{equation}

If a detector has a response function $q^+$, then the detector response $R$ to an external source $q$ in the unperturbed system is given by
\begin{equation}\label{eq:bg:pt:green168}
  R = \left<\psi,q^+\right>
    = \left<\psi^+,q\right>.
\end{equation}
In the perturbed system, the detector response $\bar{R}$ is given by
\begin{equation}\label{eq:bg:pt:r_bar}
  \bar{R} = \left<\bar{\psi},q^+\right>
          = \left<\bar{\psi}^+,q\right>.
\end{equation}

The perturbation in the detector response $\delta R$ is defined as the difference between the detector response in the perturbed system and the detector response in the unperturbed system; i.e.
\begin{equation}\label{eq:bg:pt:delta_r}
  \delta R \equiv \bar{R} - R.
\end{equation}
Similarly, the perturbation in the flux $\delta\psi$ is defined as the difference between the flux in the perturbed system and the flux in the unperturbed system; i.e.
\begin{equation}\label{eq:bg:pt:delta_psi}
  \delta \psi \equiv \bar{\psi} - \psi.
\end{equation}
Lastly, the perturbation in the transport operator $\delta H$ is defined as the difference between the perturbed transport operator and the unperturbed transport operator; i.e.
\begin{equation}\label{eq:bg:pt:delta_H}
  \delta H \equiv \bar{H} - H.
\end{equation}

$\delta R$ can thus be written as
\begin{equation}\label{eq:bg:pt:green170a}\begin{split}
  \delta R & = \left<\bar{\psi},q^+\right> - \left<\psi,q^+\right> \\
           & = \left<\delta\psi,q^+\right> \\
           & \stackrel{\includegraphics[height=\fontcharht\font`\B]{../pres-2019.08.30/thinking-face.png}}{=} \left<\psi^+,-\delta H\psi\right>
\end{split}\end{equation}
or similarly as
\begin{equation}\label{eq:bg:pt:green170b}\begin{split}
  \delta R & = \left<\bar{\psi}^+,q\right> - \left<\psi^+,q\right> \\
           & = \left<\delta\psi^+,q\right> \\
           & \stackrel{\includegraphics[height=\fontcharht\font`\B]{../pres-2019.08.30/thinking-face.png}}{=} \left<\psi,-\delta H^+\psi^+\right>.
\end{split}\end{equation}
These expressions for $\delta R$ are powerful because they depend only on the unperturbed forward and adjoint flux distributions.
Thus, the effects of many different perturbations can be calculated efficiently while only needing to calculate the forward and adjoint flux distributions once.
These expressions provide the basis for the flux-difference method of solving deep penetration problems.
