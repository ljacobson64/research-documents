% thesisdefs.tex

% This is mostly adapted from withesis.cls.  The original copyright notice for
% withesis.cls follows, preceded by two percent signs (%%):

%% withesis.cls
%% LaTeX Style file for the University of Wisconsin-Madison Thesis Format
%% Adapted from the Purdue University Thesis Format
%% Originally by Dave Kraynie
%% Edits by Darrell McCauley
%% Adapted to UW-Madison format by Eric Benedict  (Noted with <EB>)
%% Updated to LaTeX2e by Eric Benedict 24 July 00
%%
%%=============================================================================
%% Licensed under the Perl Artistic License.
%% see: http://www.ctan.org/tex-archive/help/Catalogue/licenses.artistic.html
%% for more info...
%%=============================================================================

% withesis.cls is available from CTAN.  The modifications to this file
% are also licensed under the Perl Artistic License.

% --wb, 2008

\makeatletter

\newcounter {tocpage}
\newcounter {lofpage}
\newcounter {lotpage}
\newcounter {listofheading}

\newcommand\@thesistitlemedskip{0.2in}
\newcommand\@thesistitlebigskip{0.6in}
\newcommand{\degree}[1]{\gdef\@degree{#1}}
\newcommand{\project}{\gdef\@doctype{A masters project report}}
\newcommand{\prelim}{\gdef\@doctype{A preliminary report}}
\newcommand{\thesis}{\gdef\@doctype{A thesis}}
\newcommand{\dissertation}{\gdef\@doctype{A dissertation}}
\newcommand{\department}[1]{\gdef\@department{(#1)}}

\newenvironment{titlepage}
{\@restonecolfalse\if@twocolumn\@restonecoltrue\onecolumn
\else \newpage \fi \thispagestyle{empty}
% \c@page\z@ -- deleted: count title page in thesis
}{\if@restonecol\twocolumn \else \newpage \fi}

\gdef\@degree{Doctor of Philosophy}
\gdef\@doctype{A preliminary report}  % Dissertation

\gdef\@department{(Engineering Physics)}
\gdef\@defensedate{2020-XX-XX}
\gdef\@committee{
  Paul P.H. Wilson,   Professor, Engineering Physics\\
  Douglass Henderson, Professor, Engineering Physics\\
  Jake Blanchard,     Professor, Engineering Physics\\
  Xiaopig Qian,       Professor, Mechanical Engineering\\
  Dan Negrut,         Professor, Mechanical Engineering\\
  }

\renewcommand{\maketitle}{%
  \begin{titlepage}
%-----------------------------------------------------------------------------
% -- The thesis office doesn't like thanks on title page. Put it in the
% -- acknowledgments. This is here so you don't have to change your title page
% -- when converting from report style. -> from Purdue, but I left it here since
% -- it seems compatible with UW-Madison. Eric
%-----------------------------------------------------------------------------
    \def\thanks##1{\typeout{Warning: `thanks' deleted from thesis titlepage.}}
    \let\footnotesize\small \let\footnoterule\relax \setcounter{page}{1}
    \begin{center}
      {\textbf{\expandafter\expandafter{\@title}}} \\[\@thesistitlemedskip]
       by \\[\@thesistitlemedskip]
      \@author \\[\@thesistitlebigskip]
      \@doctype\ submitted in partial fulfillment of \\
      the requirements for the degree of\\[\@thesistitlemedskip]
      \@degree \\[\@thesistitlemedskip]
      \@department \\[\@thesistitlemedskip]
      at the \\[\@thesistitlemedskip]
      UNIVERSITY OF WISCONSIN--MADISON\\[\@thesistitlemedskip]
      \@date
    \end{center}
    \hspace*{-0.7in}Date of preliminary oral examination: \@defensedate \\[\@thesistitlemedskip]
    \hspace*{-0.7in}Thesis Committee:\\
    \@committee
  \end{titlepage}
  \setcounter{footnote}{0}
  \setcounter{page}{1} %title page is NOT counted
  \let\thanks\relax
  \let\maketitle\relax \let\degree\relax \let\project\relax \let\prelim\relax
  \let\department\relax
  \gdef\@thanks{}\gdef\@degree{}\gdef\@doctype{}
  \gdef\@department{}
  %\gdef\@author{}\gdef\@title{}
}

% ------------------------------------------------------------------------------
% Abstract
% ------------------------------------------------------------------------------

% The abstract should begin with two single-spaced lines describing the author
% and title in a standard format.  After these lines comes the standard
% abstract.

\def\abstract{
  \chapter*{Abstract}
  \addcontentsline{toc}{chapter}{Abstract}
  \relax\markboth{Abstract}{Abstract}}
\def\endabstract{\par\newpage}

% ------------------------------------------------------------------------------
% UMI Abstract
% ------------------------------------------------------------------------------

% The UMI abstract should begin with the author and title in a standard format.
% After the author comes the advisor and university. After these lines comes
% a bunch of double spaced text to make up the standard abstract. After the
% abstract, the advisor's approval signature follows. This page is not numbered
% and is delivered seperately to the thesis office.

\def\advisortitle#1{\gdef\@advisortitle{#1}}
\def\advisorname#1{\gdef\@advisorname{#1}}
\gdef\@advisortitle{Professor}
\gdef\@advisorname{Cheer E.\ Place}

\def\umiabstract{
  \thispagestyle{empty}
  \addtocounter{page}{-1}
  \begin{center}
    {\textbf{\expandafter\uppercase\expandafter{\@title}}}\\
    \vspace{12pt}
    \@author \\
    \vspace{12pt}
    Under the supervision of \@advisortitle\ \@advisorname\\
    At the University of Wisconsin--Madison
  \end{center}
}

\def\endumiabstract{\vfill \hfill\@advisorname\par\newpage}

% ------------------------------------------------------------------------------
% Verbatimfile
% ------------------------------------------------------------------------------

% \verbatimfile{<filename>} for verbatim inclusion of a file
% - Note that the precise layout of line breaks in this file is important!
% - added the \singlespace - EB

\def\verbatimfile#1{
  \begingroup
  \singlespace
  \@verbatim
  \frenchspacing
  \@vobeyspaces
  \input#1
  \endgroup
}

% ------------------------------------------------------------------------------
% Separator Pages
% ------------------------------------------------------------------------------

% Creates a blank page with a text centered horizontally and vertically. The
% page is neither counted nor numbered. These pages are required in the thesis
% format before sections such as appendices, vita, bibliography, etc.

\def\separatorpage#1{
  \newpage
  \thispagestyle{empty}
  \addtocounter{page}{-1}
  \null
  \vfil\vfil
  \begin{center}
    {\textbf{#1}}
  \end{center}
  \vfil\vfil
  \newpage
}

% ------------------------------------------------------------------------------
% Copyright Page
% ------------------------------------------------------------------------------

% The copyright must do the following:
% - start a new page with no number
% - place the copyright text centered at the bottom.

\def\copyrightpage{
  \newpage
  \thispagestyle{empty}  % No page number
  \addtocounter{page}{-1}
  \chapter*{}  % Required for \vfill to work
  \begin{center}
    \vfill
    \copyright\ Copyright by \@author\ \@date\\
    All Rights Reserved
  \end{center}
}

% ------------------------------------------------------------------------------
% Glossary
% ------------------------------------------------------------------------------

% The glossary environment must do the following:
% - produce the table of contents entry for the glossary
% - start a new page with GLOSSARY centered two inches from the top

\def\glossary{
  \chapter*{GLOSSARY}
  \addcontentsline{toc}{chapter}{Glossary}
}
\def\endglossary{\par\newpage}

% ------------------------------------------------------------------------------
% Nomenclature
% ------------------------------------------------------------------------------

% The nomenclature environment must do the following:
% - produce the table of contents entry for the nomenclature section
% - start a new page with NOMENCLATURE centered two inches from the top

\def\nomenclature{
% \separatorpage{DISCARD THIS PAGE}
  \chapter*{Nomenclature}
  \addcontentsline{toc}{chapter}{Nomenclature}
}
\acsetup{first-style=long (short)}
\DeclareInstance{acro-title}{empty}{sectioning}{name-format =}

\DeclareAcronym{mcnp}{
  short = MCNP,
  long  = Monte Carlo N-Particle transport code,
  class = abbrev
}

\DeclareAcronym{advantg}{
  short = ADVANTG,
  long  = AutomateD VAriaNce reducTion Generator,
  class = abbrev
}

\def\endnomenclature{\par\newpage}

% ------------------------------------------------------------------------------
% Conventions
% ------------------------------------------------------------------------------

% The conventions environment must do the following:
% - produce the table of contents entry for the nomenclature section
% - start a new page with CONVENTIONS centered two inches from the top

\def\conventions{
  \separatorpage{DISCARD THIS PAGE}
  \chapter*{Conventions}
  \addcontentsline{toc}{chapter}{CONVENTIONS}
}
\def\endconventions{\par\newpage}

% ------------------------------------------------------------------------------
% Colophon
% ------------------------------------------------------------------------------

% The colophon environment must do the following:
% - produce the table of contents entry for the nomenclature section
% - start a new page with COLOPHON centered two inches from the top

\def\colophon{
  \separatorpage{DISCARD THIS PAGE}
  \chapter*{Colophon}
  \addcontentsline{toc}{chapter}{Colophon}
}
\def\endcolophon{\par\newpage}

% ------------------------------------------------------------------------------
% List of Symbols
% ------------------------------------------------------------------------------

% The list of symbols environment must do the following:
% - produce the table of contents entry for the list of symbols section
% - start a new page with LIST OF SYMBOLS centered two inches from the top

\def\listofsymbols{
  \separatorpage{DISCARD THIS PAGE}
  \eject
  \chapter*{LIST OF SYMBOLS}
  \addcontentsline{toc}{chapter}{LIST OF SYMBOLS}
}
\def\endlistofsymbols{\par\newpage}

% ------------------------------------------------------------------------------
% Vita
% ------------------------------------------------------------------------------

% The vita environment must do the following:
% - produce a separator page with the word vita centered
% - produce the table of contents entry for the vita
% - start a new page with VITA centered two inches from the top

\def\vita{
% \separatorpage{VITA}  % UW doesn't require this EB
  \chapter*{VITA}
  \addcontentsline{toc}{chapter}{VITA}}
\def\endvita{\par\newpage}

% ------------------------------------------------------------------------------
% ACKNOWLEDGMENTS
% ------------------------------------------------------------------------------

% The acknowledgments environment must do the following:
% - start a new page with ACKNOWLEDGMENTS centered two inches from the top

\def\acks{\chapter*{Acknowledgments}}
\def\endacks{\par\newpage}

% ------------------------------------------------------------------------------
% Dedication
% ------------------------------------------------------------------------------

% The dedication environment must do the following:
% - start a new page
% - center the text vertically
% - include the text in a center environment

\def\dedication{
  \newpage
  \null\vfil
  \begin{center}}
\def\enddedication{\end{center}\par\vfil\newpage}

% ------------------------------------------------------------------------------
% Date
% ------------------------------------------------------------------------------

%\def\today{\ifcase\month\or
  %January\or February\or March\or April\or May\or June\or
  %July\or August\or September\or October\or November\or December\fi
  %\space\number\day, \number\year}
\newcount\@testday
\def\today{
  \@testday=\day
  \ifnum\@testday>30 \advance\@testday by -30
  \else\ifnum\@testday>20 \advance\@testday by -20
  \fi\fi
  \number\day\ \
  \ifcase\month\or
    January \or February \or March \or April \or May \or June \or
    July \or August \or September \or October \or November \or December
    \fi\ \number\year
}

% ------------------------------------------------------------------------------
% Single counter for theorems and theorem-like environments
% ------------------------------------------------------------------------------

\newtheorem{theorem}{Theorem}[chapter]
\newtheorem{assertion }[theorem]{Assertion  }
\newtheorem{claim     }[theorem]{Claim      }
\newtheorem{conjecture}[theorem]{Conjecture }
\newtheorem{corollary }[theorem]{Corollary  }
\newtheorem{definition}[theorem]{Definition }
\newtheorem{example   }[theorem]{Example    }
\newtheorem{figger    }[theorem]{Figure     }
\newtheorem{lemma     }[theorem]{Lemma      }
\newtheorem{prop      }[theorem]{Proposition}
\newtheorem{remark    }[theorem]{Remark     }

\makeatother
