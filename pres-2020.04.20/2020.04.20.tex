\documentclass[t]{beamer}
\usefonttheme{serif}
\usetheme[white]{Wisconsin}
\title{Overview of the Dakota toolkit}
%\subtitle{}
\author{Lucas Jacobson}
\institute{University of Wisconsin--Madison}
\date{20 April 2020}

\usepackage{amsmath}
\usepackage{textcomp}

\newcommand{\tildecenter}{\raisebox{0.5ex}{\texttildelow}}

\begin{document}

% ==============================================================================

\begin{frame}
  \titlepage
\end{frame}

% ==============================================================================

\begin{frame}
  \frametitle{What is Dakota?}
  \begin{itemize}
    \item Robust, usable software for optimization and uncertainty
          quantification (UQ)
    \item Mainly developed at Sandia National Laboratory
    \item Released as open source under the GNU Lesser General Public License
    \item Website: https://dakota.sandia.gov
  \end{itemize}
\end{frame}

% ==============================================================================

\begin{frame}
  \frametitle{Motivation for initial development}
  \begin{itemize}
    \item Started in 1994 as an internal R\&D activity at Sandia to provide a
          common set of optimization tools for structural analysis problems
    \item Prior to Dakota, there was no effort to archive optimization methods
          for resuse on other projects
    \item Thus, engineers had to build new custom interfaces between engineering
          analysis software and optimization software for each new application
    \item Initial Dakota toolkit provided access to a variety of optimization
          algorithms, hiding the complexity of the underlying software from
          users
    \item Over the years, Dakota has grown to include methods for
    \begin{itemize}
      \item global sensitivity and variance analysis
      \item parameter estimation
      \item uncertainty quantification (UQ)
      \item verification
      \item surrogate-based optimization, hybrid optimization, and optimization
            under uncertainty
    \end{itemize}
  \end{itemize}
\end{frame}

% ==============================================================================

\begin{frame}
  \frametitle{Dakota capabilities}
  \begin{enumerate}[1]
    \item Parameter studies
    \item Design of experiments
    \item Uncertainty quantification
    \item Optimization
    \item Calibration
  \end{enumerate}
  \vskip 0.5in
  \begin{itemize}
    \item These capabilities are described in the succeeding 5 slides
    \item Note: the information in these slides is taken partially verbatim from
          the \href{https://dakota.sandia.gov/content/manuals}{Dakota Version
          6.10 User's Manual}
  \end{itemize}
\end{frame}

% ==============================================================================

\begin{frame}
  \frametitle{Parameter studies}
  \begin{itemize}
    \item Employ deterministic designs to explore the effect of parametric
          changes within simulation models, yielding one form of sensitivity
          analysis
    \item Can help assess simulation characteristics such as smoothness,
          multi-modality, robustness, and nonlinearity, which affect the choice
          of algorithms and controls in follow-on optimization and UQ studies
    \item Typical examples include centered, one-at-a-time variations or joint
          variation on a grid
  \end{itemize}
\end{frame}

% ==============================================================================

\begin{frame}
  \frametitle{Design of experiments}
  \begin{itemize}
    \item Acronym: design and analysis of computer experiments (DACE) techniques
    \item Used to explore the parameter space of an engineering design problem,
          for example to perform global sensitivity analysis
    \item Can help reach conclusions similar to parameter studies, but the
          primary goal of these methods is to generate good coverage of the
          input parameter space
    \item Representative methods include Latin hypercube sampling, orthogonal
          arrays, and Box-Behnken designs
  \end{itemize}
\end{frame}

% ==============================================================================

\begin{frame}
  \frametitle{Uncertainty quantification}
  \begin{itemize}
    \item Also referred to as nondeterministic analysis methods
    \item Compute probabilistic information about response functions based on
          simulations performed according to specified input parameter
          probability distributions
    \item Perform a forward uncertainty propagation in which probability
          information for input parameters is mapped to probability information
          for output response functions
    \item Approaches include Monte Carlo sampling, reliability methods, and
          polynomial chaos expansions
  \end{itemize}
\end{frame}

% ==============================================================================

\begin{frame}
  \frametitle{Optimization}
  \begin{itemize}
    \item Minimize cost or maximize system performance, as predicted by the
          simulation model, subject to constraints on input variables or
          secondary simulation responses
    \item Categories of algorithms include gradient-based, derivative-free, and
          global optimization
    \item Dakota also includes capabilities for multi-objective trade-off
          optimization and automatic scaling of problem formulations
    \item Advanced Dakota approaches include hybrid (multi-method), multi-start
          local, and Pareto-set optimization
  \end{itemize}
\end{frame}

% ==============================================================================

\begin{frame}
  \frametitle{Calibration}
  \begin{itemize}
    \item Maximize agreement between simulation outputs and experimental data
          (or desired outputs)
    \item Solve inverse problems (often referred to as parameter estimation or
          least-squares problems)
    \item Dakota approaches include nonlinear least squares and Bayesian
          calibration
  \end{itemize}
\end{frame}

\end{document}
